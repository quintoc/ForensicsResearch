\section{Introduction}
The National Institute of Standards and Technology (NIST) has a subdivision named the Computer Forensics Tool Testing Program (CFTT). 
The CFTT has standards regarding digital forensic tools to help determine the quality and integrity of these tools. The integrity 
of such tools is important in many cases such as corporate, security, and especially in judicial proceedings. 
Using a software that does not follow these standards has the potential to recover files incorrectly, corrupted, or 
mislabeled, which has the potential to throw out evidence in court cases that are increasingly reliable on digital data. 
On the other hand, incorrectly recovered files can also lead to jail sentence of innocent defendants. One important 
task in digital forensics is deleted file recovery (DFR), which is the focus of our proposed research. 
CFTT standards for DFR tools consists of 4 core features and a set of \emph{optional} features. 
We have already done a few preliminary experiments with the popular digital forensics tool Autopsy (The SluethKit), which 
shows importance of a similar study in a bigger scale.

\section{Research Topic}
Define what does a deleted file recovery tool do. I am interested in investigating which software meet the standards 
set by CFTT. There are many companies and individuals marketing their software as the best recovery tool. My 
research question encompasses this software as well to test their effectiveness at recovering files compared to a standard for enterprise 
level tools. I aim to use Autopsy as a base comparison to the other software and what they can do. The question 
also expands to the different types of file systems. While there are 3 operating systems that have different file 
systems, there are even more file systems that each store data and meta data differently. Looking out for Type I and 
Type II errors are also a large part of evaluating each software. Many other factors also play a part, including the 
condition of the filesystem (is it full, was the tool installed after the file was deleted, etc.) and how the file was 
deleted (Recycle bin, permanent delete, reformat of disk etc.). 
All of these additional variables can be tested and compared, especially using 
the optional requirements of the CFTT to create a comprehensive comparison for readers at both the consumer and student levels.
